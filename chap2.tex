\chapter{The Art of Combinatorial Thinking}
\section{Binomial Coefficients}
This sections aims to push-forward our familiarity with binomial coefficients which we have gained from the previous chapter. We do so by outlining proofs of a few standard combinatorial identities. 

\begin{comment}
\begin{question}
3a
\end{question}
You have $2n$ objects
Divide the objects into two classes
Each class has  $n$ objects
Choose  $k$ objetcs from class  $A$
Choose  $n-k$ objects from class  $B$
Choose a distinguished element from the k choices of objects from class  $A$
==
Choose the distinguished element from $A$ first <=> n
Choose the rest of the elements from  $A\cup B \backslash \{x\}$ where  $x$ is the distinguised elements <=> \binom{2n-1}{n-1}

\begin{question}
3b
\end{question}
Think along subsets

\begin{question}
\sum_{i=1}^{n}i(n-1) = \\sum_{i=1}^{n}\binom{i}{2} = \sum_{i=0}^{n-2}\binom{n-i}{2} = \binom{n+1}{3} 
\end{question}
Three numbers out of the $n+1$ choices say $a_{1}<a_{2}<a_{3}$. Now argue for the first three sums based on which one of these is chosen first. 

The first sum corresponds to choosing $a_2$. The second sum corresponds to choosing  $a_{3}$. The third sum corresponds to choosing $a_{1}$.

\begin{question}
	For all $n\geq r\geq 0$ we have 
	 \[
		 \binom{r}{r}+\binom{r+1}{r}+\cdots+\binom{n}{r} = \binom{n+1}{r+1}
	.\] 
\end{question}
Give a proof using lattice paths. 
RHS is the number of lattice paths from $(0,0)\to (r+1,n-r)$. For the LHS what happens before the last E step.

The idea is not always to prove equalities. Infact more often than not we will be interested in proving inequalities
 \begin{question}
	 \binom{2n}{n}<4^n
\end{question}
Convert binary to lattice: 0 to N, 1 to E and so on.

Next we prove the Chu-Vandermonde identity.
\begin{question}
	\binom{2n}{n} = \sum_{k=0}^{n} \binom{n}{k}^2
\end{question}

\begin{question}
	For all $n>0$. \[
		\sum_{k=0}^{n} 2^k\binom{n}{k} = 3^n
	\]
\end{question}

\begin{question}
	For $n>0$  \[
		2\binom{2n-1}{n} = \binom{2n}{n}
	.\] 
\end{question}
Give two proofs. One using Pascals identity. The other by choosing sets.
\end{comment}

\begin{claim}
\[
\sum_{k=1}^{n}k\binom{n}{k}^2 = n\binom{2n-1}{n-1}
\]
\end{claim}
\begin{proof}
Let $J$ be a collection of $2n$ objects which is partitioned into $2$ equal sub-collections $J_1$ and $J_2$. Now, we want to make a choice of $n$ elements from $J$ which includes a distinguished element, say $x$ from $J_1$ (or equivalently $J_2$). Since there are $n$ ways to choose the said distinguished element, and $\binom{2n-1}{n-1}$ ways to choose the rest of the $n-1$ elements, in all we have \[
n\binom{2n-1}{n-1}
\] ways to make the said choice. Alternatively, we can also choose $k$ elements from $J_1$ in $\binom{n}{k}$ ways, $n-k$ elements from $J_2$ in $\binom{n}{n-k}$ ways, and then choose the distinguished element from $J_1$ (or equivalently $J_2$) in $1\leq k\leq n$ ways. Thus far, we have proved
\[
\sum_{k=1}^{n}k\binom{n}{k}\binom{n}{n-k} = n\binom{2n-1}{n-1}.
\]
Since by \cref{r:1.2} \[
\binom{n}{k} = \binom{n}{n-k},
\]
we are done. 
\end{proof}
\begin{claim}
\[
\sum_{k=1}^{n}k\binom{n}{k} = n2^{n-1}
\]
\end{claim}
\begin{proof}
\end{proof}

\begin{claim}
\[
\sum_{i=1}^{n}i(n-1) = \sum_{i=1}^{n}\binom{i}{2} = \sum_{i=0}^{n-2}\binom{n-i}{2} = \binom{n+1}{3} 
\]
\end{claim}
\begin{proof}
\end{proof}

\begin{claim}
For all $n\geq r\geq 0$ we have 
\[
		 \binom{r}{r}+\binom{r+1}{r}+\cdots+\binom{n}{r} = \binom{n+1}{r+1}
.\] 
\end{claim}
\begin{proof}
\end{proof}

\begin{claim}
$\binom{2n}{n}<4^n$
\end{claim}
\begin{proof}
    
\end{proof}

\begin{claim}[Chu-Vandermonde Identity]
\[\binom{2n}{n} = \sum_{k=0}^{n} \binom{n}{k}^2\]
\end{claim}
\begin{proof}
Once again, by \cref{r:1.2}, notice how \[
\sum_{k=1}^n\binom{n}{k}^2 = \sum_{k=1}^{n}\binom{n}{k}\binom{n}{n-k}.
\] Now we argue combinatorially. Let $J$ be a collection of $2n$ objects which is partitioned into $2$ equal sub-collections $J_1$ and $J_2$. Any choice of $n$ elements from $J$ involves choosing $k$ elements from $J_1$ and $n-k$ elements from $J_2$ where $0\leq k\leq n$. 
\end{proof}
\begin{claim}
	For all $n>0$. \[
		\sum_{k=0}^{n} 2^k\binom{n}{k} = 3^n
	\]
\end{claim}
\begin{proof}
\end{proof}
\begin{claim}
For all $n>0$  \[
2\binom{2n-1}{n} = \binom{2n}{n}
.\] 
\end{claim}
\begin{proof}
\end{proof}
\section{Catalan Numbers}
\endinput
