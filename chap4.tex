\chapter{Partitions}
A first course in combinatorics typically focuses on two types of partitions: set partitions and integer partitions. We will begin with a brief discussion of set partitions, followed by a more in-depth exploration of integer partitions. 
\section{Set Partitions}
\textcolor{red}{
[This section is a work in progress!!!] Note to scribe:
\begin{enumerate}
    \item Finish proofs of Claim 1.1-1.6.
\end{enumerate}
}
Notice how there are $6$ ways to partition the set $\{1,2,3,4\}$ into $3$ blocks. These are,
    \begin{enumerate}
        \item $[1],[2],[3,4]$
        \item $[1],[2,3],[4]$
        \item $[1,2],[3],[4]$
        \item $[1,4],[2],[3]$
        \item $[1,3],[2],[4]$
        \item $[2,4],[1],[3]$
    \end{enumerate}
This kind of counting is generalized by what are called a Stirling numbers of the $\nth{2}$ kind. More formally,
\begin{definition}
A set partition of a finite set $B$ into $k$ ``blocks'' is a collection of $k$ subsets of $B$ say $B_1,\cdots,B_k$ such that 
\begin{enumerate}
    \item $\bigcup_{i=1}^kB_i = B$,
    \item $B_i\cap B_j=\emptyset$ for all $i\neq j$,
    \item and none of the $B_i$'s are empty. 
\end{enumerate}
\end{definition}
\begin{definition}
If $B=[n]=\{1,\cdots,n\}$ then a Stirling number of the $\nth{2}$ kind $S(n,k)$, is the number of set partitions of $B$ into $k$ blocks. 
\end{definition}
We take $S(0,0)$ to be $1$ by convention. Additionally, the fact that $S(n,n)=1$, $S(n,0)=0$, and $S(n,1)=1$ follow immediately. Per usual we state a few identities concerning these numbers.
\begin{claim}
For all $n,k\geq 0$, with $n\geq k$ we have $S(n,k)=S(n-1,k-1)+kS(n-1,k)$
\end{claim}
\begin{claim}
For all $n,k\geq 0$, with $n\geq k$ we have $S(n+1,k)=\sum_{i=0}^{n}S(n-i,k-1)$
\end{claim}
\begin{claim}
    $S(n,2)=2^{n-1}-1$
\end{claim}
\begin{claim}
    $S(n,n-1)=\binom{n}{2}$
\end{claim}
\begin{claim}
    \[
    S(n,k) = \dfrac{1}{k!}\sum_{j=0}^{k}(-1)^{k-j}\binom{k}{j}j^n
    \]
\end{claim}
\begin{comment}
\begin{proof}
Recall how $\sum_{j=0}^{k}(-1)^{k-j}\binom{k}{j}j^n$ counts the number of surjections from $[n]$ to $[k]$. Hence, it suffices to show that the number of surjections from $[n]$ to $[k]$ is $k!S(n,k)$. 
\end{proof}
\end{comment}
\begin{definition}[Bell Numbers]
The number of all partitions of $[n]$ is called a bell number and is denoted by $B(n)$. More specifically,
\[
B(n) = \sum_{k=1}^nS(n,k)
\]
\end{definition}
We are interested in coming up with a recurrence of $B(n)$ which is independent of any $S(n,k)$s. 
\begin{claim}
    \[
    B(n+1) = \sum_{k=0}^{n}\binom{n}{n-k}B(k)
    \]
\end{claim}
\section{Integer Partitions}
\textcolor{red}{
[This section is a work in progress!!!] Note to scribe: Add the following topics
\begin{enumerate}
    \item Finish details of the proof of EPT
    \item JTP
    \item EPT as a consequence of JTP
    \item $q$-Pochammer symbols and a few re-statements
\end{enumerate}
}
Recall how with \cref{d:1.5} we defined integer partitions. The study of these partitions dates back to the work of Leonard Euler in the \nth{18} century, who introduced generating functions to analyze them. Their study was then developed further through the work of mathematicians like Srinivasa Ramanujan and Major MacMachon, who revealed deep arithmetic and combinatorial properties. For instance, if $p(n)$ denotes the number of partitions of $n$, then the celebrated Ramanujan congruences state that
\begin{align*}
P(5n+4)&\equiv 0\mod{(5)} \\
P(7n+5)&\equiv 0\mod{(7)} \\
P(11n+6)&\equiv 0\mod{(11)} \\ 
\end{align*}
In the absence of a closed form (recall \cref{r:1.3}), we are interested in finding the generating function of $p(n)$.
\begin{claim}\[
\prod_{n\geq 1}(1+q^n+q^{2n}+\cdots) = \sum_{n\geq 0}p(n)q^n.
\]
\label{c:2.1P}
\end{claim}
\begin{proof}
We can expand the product on the L.H.S, \[(1+q+q^2+q^3+\cdots)(1+q^2+q^4+q^6+\cdots)(1+q^3+q^6+q^9+\cdots)\cdots,\] out by choosing one term from each factor in all possible ways. If we then collect like terms, the coefficient of $q^k$ will be the number of ways to choose one term from each factor so that the exponents of the said terms sum up to $k$. This is also what the R.H.S counts. For instance $q^3$ can be obtained in the following ways
\begin{enumerate}
    \item Choose $q^3$ from the first bracket and $1$ from every other bracket. This corresponds to the partition $3=1+1+1$.
    \item Choose $q$ from the first bracket, $q^2$ from the second bracket, and $1$ from every other bracket. This corresponds to the partition $3=2+1$. 
    \item Chose $1$ from the first two brackets and $q^3$ from the third bracket. This corresponds to the partition $3=3$. 
\end{enumerate}
\end{proof}
Next, we are interested in looking at a two refinements of partitions which follow quite naturally by the arguments we used in the previous proof.
\begin{claim}
    Let $p_E(n)$ and $p_O(n)$ denote the number of partitions of $n$ into even and odd parts respectively. Then,
    \begin{align*}
    \prod_{n=1,3,5,\cdots}(1+q^n+q^{2n}+\cdots)=\sum_{n\geq 0}p_O(n)q^n \\
    \prod_{n=2,4,6,\cdots}(1+q^n+q^{2n}+\cdots)=\sum_{n\geq 0}p_E(n)q^n 
    \end{align*}
    \label{c:PEPO}
\end{claim}
Next, we state an interesting result, once again due to Euler.
\begin{theorem}[Euler's Gem]
The number of partitions of $n$ into distinct parts, say $p_D(n)$, is the same as the number of partitions of $n$ into odd parts, say $p_O(n)$.
\label{t:Euler'sGem}
\end{theorem}
We will outline two different proofs of the result.
\begin{proof}
By \cref{c:PEPO} we already know that the generating function for $p_O(n)$ is 
\[
\prod_{i=1}^{\infty}\dfrac{1}{1-q^{2i-1}}.
\]
It is also clear (by arguments similar to the ones which were involved in the proof of $\cref{c:2.1P}$) that the generating function for $p_D(n)$ is 
\[
(1+q)(1+q^2)(1+q^3)\cdots = \prod_{i=1}^\infty(1+q^i).
\]
Now to complete the proof it suffices to show that the two are equal. To this end, notice how 
\begin{align*}
    \prod_{i=1}^\infty(1+q^i) &= \prod_{i=1}^\infty (1+q^i)\dfrac{1-q^i}{1-q^i} \\
    &= \prod_{i=1}^\infty \dfrac{1-q^{2i}}{1-q^i} \\\
    &= \dfrac{(1-q^2)(1-q^4)(1-q^6)\cdots}{(1-q)(1-q^2)(1-q^3)(1-q^4)\cdots} \\
    &= \prod_{i=1}^\infty \dfrac{1}{1-q^{2i-1}}
\end{align*}
\end{proof}
In the spirit of giving a combinatorial proof, we want to setup a bijection called Glashier's bijection between the two types of partitions.
\begin{proof}
First we setup a map which sends a partition into odd parts to a partition into distinct parts. The most natural thing to do is to merge any pairs of repeating parts into one part of double the size. We can repeat this procedure until all the parts are distinct. For instance $3+3+3+1+1+1+1\to (3+3)+3+(1+1)+(1+1)= 6+3+2+2 = 6+3+(2+2) = 6+4+3$. Next, we setup a map which sends a partition into distinct parts to a partition into odd parts. Once again, the most natural thing to do is split every occurrence of an even part into two equal parts. We can repeat this procedure until all the parts are odd. For instance $6+4+3\to (3+3)+(2+2)+3\to 3+3+3+1+1+1+1$.
\end{proof}
We state a generalization of Euler's gem now. 
\begin{theorem}
The number of partitions where no part appears $d$ or more times is the same as the number of partitions where no part is divisible by $d$.
\label{t:glashier}
\end{theorem}
\begin{proof}
Let $p_1(n)$ denote the number of partitions of $n$ with no parts divisible by $d$. Let $p_2(n)$ denote the number of partitions of $n$ where no part appears $d$ or more times. Notice how the generating function for $p_1(n)$ is given by \[
\sum_{n=0}^{\infty}p_1(n)q^n = \prod_{n=1, d\nmid n}^{\infty}\dfrac{1}{1-q^n}
\]
and that of $p_2(n)$ is given by 
\begin{align*}
\sum_{n=0}^{\infty}p_2(n)q^n &= \prod_{n=1}^{\infty}\dfrac{1-q^{dn}}{1-q^n} \\
&= \dfrac{1-q^d}{1-q}\dfrac{1-q^{2d}}{1-q^2}\cdots\dfrac{1-q^{kd}}{1-q^k}\cdots.
\end{align*}
Finally, each term in the numerator cancels with the corresponding multiple of $d$ in the denominator and we are left with the generating function for $p_1(n)$. This completes the proof.
\end{proof}
\begin{remark}
One might ask in what sense is \cref{t:glashier} a generalization of \cref{t:Euler'sGem} To see this, notice how setting $d=2$ returns the Euler's gem.  
\end{remark}
A rather interesting corollary of \cref{t:glashier} can be used to see why the binary representation of a number is unique. Consider the set of all partitions of $n$ which has parts only of size $1$. Clearly, the only such partition is $\underbrace{1+\cdots+1}_{n\text{ times}}$. Next, we are interested in merging the said parts to the point where the resultant partition has all parts distinct. It is also clear that any such sequence of merges will result in a partition where the parts occur as powers of $2$. On the other hand, any partition which only has powers of $2$ as it's parts can be repeatedly split repeatedly to the point where the resultant partition has only $1$s in it. This completes a neat proof of the uniqueness of a binary representation. 
\par
Although the proof of Euler's Gem is fairly straightforward, it is not clear how one might come up with such an identity. To this end, we shall try and guess an identity due to Leonard Rogers and Srinivasa Ramanujan. Consider those partitions of $1\leq n\leq 10$ which have parts differing by atleast $2$. We list these in a table below.
\begin{table}[H]
\centering
\begin{tabular}{|c|c|l|}
\hline
$n$ & \# & Admissible partitions of $n$ \\
\hline
1 & 1 & 1 \\
2 & 1 & 2 \\
3 & 1 & 3 \\
4 & 2 & 4, 3+1 \\
5 & 2 & 5, 4+1 \\
6 & 3 & 6, 5+1, 4+2 \\
7 & 3 & 7, 6+1, 5+2 \\
8 & 4 & 8, 7+1, 6+2, 5+3 \\
9 & 5 & 9, 8+1, 7+2, 6+3, 5+3+1 \\
10 & 6 & 10, 9+1, 8+2, 7+3, 6+4, 6+3+1 \\
\hline
\end{tabular}
\caption{Partitions of $n$ into what are called $2$-distinct parts.}
\label{table:1}
\end{table}
\raggedbottom
Next, we attempt to construct a set $X$ such that the number of partitions of $n$ with parts in $X$ are the same as the one we have counted in \cref{table:1}. To this end, we make a series of observations.
\begin{enumerate}
    \item Since there should be one partition of $1$ with parts in $X$, $1$ must necessarily be a part of $X$.
    \item Since there should be one partition of $2$ with parts in $X$, $1$ is already in $X$, and $2=1+1$, we need not add $2$ to $X$. Similarly, we have no additions to $X$ corresponding to $3$ either.
    \item Since there should be two partitions of $4$ with parts in $X$, $1$ is in $X$, and $4=1+1+1+1$, we need one more partition. Hence, we add $4$ to $X$ to take care of the partition $4=4$. 
    \item Since there should be two partitions of $5$ with parts in $X$, $1$ and $4$ are in $X$, and $5=1+1+1+1+1=1+4$, we need not add $5$ to $X$.
    \item $\cdots$
\end{enumerate}
Doing this exercise for the remaining numbers ($6\leq n\leq 10$) makes it clear that we must also add $6$ and $9$ must also be added to $X$. A pattern presents itself, namely, that all the members have admit $1$ or $4$ as a remainder upon division by $5$. In summary, we have guessed (not proved) that the number of partitions of $n$ into $2$-distinct parts is the same as the number of partitions of $n$ into parts which are congruent to $1,4\mod{5}$.  
\par
As it turns out many partition identities are best explained using pictures. To this end we introduce a tool called Ferrer's diagrams. In the said diagram the parts of a partition are shown as rows of dots/squares. 
\begin{figure}[H]
    \centering
    \includegraphics[width=0.5\linewidth]{Images/Figure17.png}
    \caption{The Ferrer's diagram for $7+5+3+2+2+1$}
\end{figure}
It is clear that height of the Ferrer's diagram corresponds to the number of parts in the partition and that the width corresponds to the size of the largest part in the partition. To this end, consider the following result. 
\begin{theorem}
The  number of partitions of $n$ into atmost $i$ parts, each of which is atmost $j$ is the same as the number of partitions of $n$ into atmost $j$ parts, each of which is atmost $i$. 
\end{theorem}
\begin{proof}
Let $\pi$ a partition of $n$ into atmost $i$ parts each of which is atmost $j$. The Ferrer's diagram corresponding to $\pi$ now has height atmost $i$ and width atmost $j$. Flipping the said diagram about it's main diagonal results in a diagram which has height atmost $j$ and width atmost $i$. Finally, the partition corresponding to this flipped diagram (called the conjugate partition) is a partition of $n$ into atmost $j$ parts, each of which is atmost $i$. 
\end{proof}
\begin{figure}[H]
    \centering
    \includegraphics[width=0.8\linewidth]{Images/Figure18.png}
    \caption{An example of the conjugate of $7+5+3+2+2+1$}
\end{figure}
Another useful pictorial construct is that of a Durfee square. We define the said square to be the largest one which can fit in the Ferrer’s diagram corresponding to a partition.
\begin{figure}[H]
    \centering
    \includegraphics[width=0.5\linewidth]{Images/Figure19.png}
    \caption{The Durfee square of order $3$ in the partition $7+5+3+2+2+1$}
    \label{DurfeeExample}
\end{figure}
(Keeping in mind \cref{DurfeeExample}) notice how
\begin{enumerate}
    \item Every non-empty partition has a Durfee square of order $k$ (if nothing, you always have the Durfee square of order $k=1$).
    \item Partitions corresponding to the triangle above the Durfee square are the ones where each part is atmost $k$.
    \item Partitions corresponding to the triangle below the Durfee square are the ones where number of parts are atmost $k$. 
\end{enumerate}
With these three observations at hand, an identity presents itself almost immediately. Namely,
\begin{claim}[Jacobi's Identity]
    \[
    \prod_{i=1}^{\infty}\dfrac{1}{1-q^i} = \sum_{i=0}^{\infty}\dfrac{q^{i^2}}{(1-q)^2(1-q^2)^2\cdots(1-q^i)^2}
    \]
\end{claim}
Next, we state a remarkable result first proved by Euler in the year 1785.
\begin{theorem}[Euler's Pentagonal Number Theorem]
\begin{align*}
\prod_{i=1}^{\infty}(1-q^i) &= \sum_{k=-\infty}^{\infty}(-1)^k q^{\dfrac{k(3k-1)}{2}} \\
&= 1+\sum_{k=1}^{\infty}(-1)^kq^{\dfrac{k(3k-1)}{2}}+\sum_{k=-\infty}^{-1}(-1)^kq^{\dfrac{k(3k-1)}{2}} \\
&= 1+\sum_{k=1}^{\infty}(-1)^kq^{\dfrac{k(3k-1)}{2}}+\sum_{k=1}^{\infty}(-1)^kq^{\dfrac{k(3k+1)}{2}} \\
&= 1+\sum_{k=1}^{\infty}(-1)^k \left(q^{\dfrac{k(3k-1)}{2}}+q^{\dfrac{k(3k+1)}{2}}\right) \\
\end{align*}
\label{t:EPT}
\end{theorem}
We are interested in stating three different proofs of the theorem. Before moving on to them, it would help to understand what pentagonal numbers are. Simply put, these are numbers of the form $n(3n-1)/2$. Why they are called pentagonal is clear from the following figure.
\begin{figure}[H]
    \centering
    \includegraphics[width=0.8\linewidth]{Images/Figure20.png}
    \caption{Counting the number of vertices (marked in red) at each iteration gives the sequence of pentagonal numbers, i.e, $1,5,12,\ldots$}
    \label{f:PNT}
\end{figure}
Notice how the vertices (marked in red) in \cref{f:PNT} admits a natural partition and hence a Ferrer's diagram. 
\begin{figure}[H]
    \centering
    \includegraphics[width=0.8\linewidth]{Images/Figure21.png}
    \caption{The first $3$ pentagonal numbers and the natural partitions they admit, i.e, $1=1, 5=3+2$ and $12=5+4+3$.}
\end{figure}
It is clear that these partitions of pentagonal numbers we have obtained are partitions into distinct parts. To this end, we will try and set up a bijection between partitions of $n$ into an odd number of distinct parts and the partitions of $n$ into an even number of distinct parts. To this end, it suffices to set-up a bijection which changes the number of partitions by exactly $1$ whilst keeping the distinctness intact. \textcolor{red}{[ADD THE MAP (Franklin's Bijection) HERE.]} In summary, if $p_{ED}(n)$ and $p_{OD}(n)$ denote
the number of partitions into an even number of distinct parts and an odd number of distinct parts respectively, then we have shown that
\begin{align*}
    p_{ED}(n)-p_{OD}(n) &= 
    \begin{cases}
        (-1)^j & \quad \text{if } n \text{ is a pentagonal number} \\
        0 & \quad \text{otherwise}
    \end{cases}.
\end{align*}
This proves \cref{t:EPT} (\textcolor{red}{[EXPLAIN HOW]}). 

\par
To see the usefulness of \cref{t:EPT} consider the following remark.
\begin{remark}
Putting \cref{t:EPT} and \cref{c:2.1P} together gives us \[
\left(\sum_{n\geq 0}p(n)q^n\right)\left(\prod_{i\geq 1}(1-q^i)\right) = 1.
\] A recurrence of sorts presents itself. Namely,
\[
p(n) = p(n-1)+p(n-2)-p(n-5)-p(n-7)+p(n-12)+p(n-15)+\cdots. 
\] The reason this recurrence is useful is two-pronged. One, the sum on the right is not infinite because $p(k)=0$ for all choices of $k<0$. Two, to compute $p(n)$ we need not compute $p(n-i)$ for all choices of $1\leq i\leq n-1$.
\end{remark}
Next, we present a result known as Jacobi's triple product identity. Interestingly, this identity is easier to prove than Euler's pentagonal number theorem. Not only that, but, Euler's theorem emerges as a corollary of Jacobi's identity. Before stating the identity, we introduce some useful notation. For formal variables $x,q$ and a natural number $n>0$ we define
\[
(x;q)_n := (1-x)(1-xq)(1-xq^2)\cdots(1-xq^{n-1}) = \prod_{i=0}^{n-1}(1-xq^i).
\]
The symbol $(x;q)_n$ is referred to as the $q$-shifted factorial. We extend our notation to allow for $n$ to be $\infty$ by defining
\[
(x;q)_\infty := \lim_{n\to \infty}\left(\prod_{i=0}^{n-1}(1-xq^i)\right) = \prod_{i=0}^{\infty}(1-xq^i) = (1-x)(1-xq)(1-xq^2)\cdots.
\]
\endinput
