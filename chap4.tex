\chapter{Partitions}
A first course in combinatorics typically focuses on two types of partitions: set partitions and integer partitions. We will begin with a brief discussion of set partitions, followed by a more in-depth exploration of integer partitions. 
\section{Set Partitions}
\textcolor{red}{
[This section is a work in progress!!!] Note to scribe:
\begin{enumerate}
    \item Finish proofs of Claim 1.1-1.6.
\end{enumerate}
}
Notice how there are $6$ ways to partition the set $\{1,2,3,4\}$ into $3$ blocks. These are,
    \begin{enumerate}
        \item $[1],[2],[3,4]$
        \item $[1],[2,3],[4]$
        \item $[1,2],[3],[4]$
        \item $[1,4],[2],[3]$
        \item $[1,3],[2],[4]$
        \item $[2,4],[1],[3]$
    \end{enumerate}
This kind of counting is generalized by what are called a Stirling numbers of the $\nth{2}$ kind. More formally,
\begin{definition}
A set partition of a finite set $B$ into $k$ ``blocks'' is a collection of $k$ subsets of $B$ say $B_1,\cdots,B_k$ such that 
\begin{enumerate}
    \item $\bigcup_{i=1}^kB_i = B$,
    \item $B_i\cap B_j=\emptyset$ for all $i\neq j$,
    \item and none of the $B_i$'s are empty. 
\end{enumerate}
\end{definition}
\begin{definition}
If $B=[n]=\{1,\cdots,n\}$ then a Stirling number of the $\nth{2}$ kind $S(n,k)$, is the number of set partitions of $B$ into $k$ blocks. 
\end{definition}
We take $S(0,0)$ to be $1$ by convention. Additionally, the fact that $S(n,n)=1$, $S(n,0)=0$, and $S(n,1)=1$ follow immediately. Per usual we state a few identities concerning these numbers.
\begin{claim}
For all $n,k\geq 0$, with $n\geq k$ we have $S(n,k)=S(n-1,k-1)+kS(n-1,k)$
\end{claim}
\begin{claim}
For all $n,k\geq 0$, with $n\geq k$ we have $S(n+1,k)=\sum_{i=0}^{n}S(n-i,k-1)$
\end{claim}
\begin{claim}
    $S(n,2)=2^{n-1}-1$
\end{claim}
\begin{claim}
    $S(n,n-1)=\binom{n}{2}$
\end{claim}
\begin{claim}
    \[
    S(n,k) = \dfrac{1}{k!}\sum_{j=0}^{k}(-1)^{k-j}\binom{k}{j}j^n
    \]
\end{claim}
\begin{comment}
\begin{proof}
Recall how $\sum_{j=0}^{k}(-1)^{k-j}\binom{k}{j}j^n$ counts the number of surjections from $[n]$ to $[k]$. Hence, it suffices to show that the number of surjections from $[n]$ to $[k]$ is $k!S(n,k)$. 
\end{proof}
\end{comment}
\begin{definition}[Bell Numbers]
The number of all partitions of $[n]$ is called a bell number and is denoted by $B(n)$. More specifically,
\[
B(n) = \sum_{k=1}^nS(n,k)
\]
\end{definition}
We are interested in coming up with a recurrence of $B(n)$ which is independent of any $S(n,k)$s. 
\begin{claim}
    \[
    B(n+1) = \sum_{k=0}^{n}\binom{n}{n-k}B(k)
    \]
\end{claim}
\section{Integer Partitions}
Recall how with \cref{d:1.5} we defined integer partitions. The study of these partitions dates back to the work of Leonard Euler in the \nth{18} century, who introduced generating functions to analyze them. Their study was then developed further through the work of mathematicians like Srinivasa Ramanujan and Major MacMachon, who revealed deep arithmetic and combinatorial properties. For instance, if $p(n)$ denotes the number of partitions of $n$, then the celebrated Ramanujan congruences state that
\begin{align*}
P(5n+4)&\equiv 0\mod{(5)} \\
P(7n+5)&\equiv 0\mod{(7)} \\
P(11n+6)&\equiv 0\mod{(11)} \\ 
\end{align*}
In the absence of a closed form (recall \cref{r:1.3}), we are interested in finding the generating function of $p(n)$.
\begin{claim}\[
\prod_{n\geq 1}(1+q^n+q^{2n}+\cdots) = \sum_{n\geq 0}p(n)q^n.
\]
\end{claim}
\begin{proof}
We can expand the product on the L.H.S, \[(1+q+q^2+q^3+\cdots)(1+q^2+q^4+q^6+\cdots)(1+q^3+q^6+q^9+\cdots)\cdots,\] out by choosing one term from each factor in all possible ways. If we then collect like terms, the coefficient of $q^k$ will be the number of ways to choose one term from each factor so that the exponents of the said terms sum up to $k$. This is also what the R.H.S counts. For instance $q^3$ can be obtained in the following ways
\begin{enumerate}
    \item Choose $q^3$ from the first bracket and $1$ from every other bracket. This corresponds to the partition $3=1+1+1$.
    \item Choose $q$ from the first bracket, $q^2$ from the second bracket, and $1$ from every other bracket. This corresponds to the partition $3=2+1$. 
    \item Chose $1$ from the first two brackets and $q^3$ from the third bracket. This corresponds to the partition $3=3$. 
\end{enumerate}
\end{proof}
Next, we are interested in looking at a two refinements of partitions which follow quite naturally by the arguments we used in the previous proof.
\begin{claim}
    Let $p_E(n)$ and $p_O(n)$ denote the number of partitions of $n$ into even and odd parts respectively. Then,
    \begin{align*}
    \prod_{n=1,3,5,\cdots}(1+q^n+q^{2n}+\cdots)=\sum_{n\geq 0}p_O(n)q^n \\
    \prod_{n=2,4,6,\cdots}(1+q^n+q^{2n}+\cdots)=\sum_{n\geq 0}p_E(n)q^n 
    \end{align*}
\end{claim}
%Show Euler's Gem
\endinput
