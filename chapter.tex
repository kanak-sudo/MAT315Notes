%-----------------------------------------------------------------------
% Beginning of chapter.tex
%-----------------------------------------------------------------------
%
%  This is a sample file for use with AMS-LaTeX.  It provides an example
%  of how to set up a file for a book to be typeset with AMS-LaTeX.
%
%  This is the driver file.  Separate chapters should be included at
%  the end of this file.
%
%  ***** DO NOT USE THIS FILE AS A STARTER FOR YOUR BOOK. *****
%  Follow the guidelines in the file chapter.template.
%
%%%%%%%%%%%%%%%%%%%%%%%%%%%%%%%%%%%%%%%%%%%%%%%%%%%%%%%%%%%%%%%%%%%%%%%%

\documentclass{amsbook}

\includeonly{preface,chap1,chap2,chap3,chap4,chap5,chap6,biblio,index}

\newtheorem{theorem}{Theorem}[chapter]
\newtheorem{lemma}[theorem]{Lemma}

\theoremstyle{definition}
\newtheorem{definition}[theorem]{Definition}
\newtheorem{example}[theorem]{Example}
\newtheorem{xca}[theorem]{Exercise}

\theoremstyle{remark}
\newtheorem{remark}[theorem]{Remark}

\numberwithin{section}{chapter}
\numberwithin{equation}{chapter}

%    Absolute value notation
\newcommand{\abs}[1]{\lvert#1\rvert}

%    Blank box placeholder for figures (to avoid requiring any
%    particular graphics capabilities for printing this document).
\newcommand{\blankbox}[2]{%
  \parbox{\columnwidth}{\centering
%    Set fboxsep to 0 so that the actual size of the box will match the
%    given measurements more closely.
    \setlength{\fboxsep}{0pt}%
    \fbox{\raisebox{0pt}[#2]{\hspace{#1}}}%
  }%
}

\begin{document}
\frontmatter
\title{MAT315 Combinatorial Enumeration\\Monsoon 2024}


\author{Instructor: Manjil Saikia}

\address{Mathematical and Physical Sciences division, School of Arts and Sciences, Ahmedabad University, Ahmedabad 380009, Gujarat, India}

\email{manjil.saikia@ahduni.edu.in}
%    \thanks will become a 1st page footnote.
%\thanks{The first author was supported in part by NSF Grant \#000000.}


\author{TA: Kanak Dhotre}
%\email{kanak.d@ahduni.edu.in}

\date{July 22, 2024}
\subjclass[2020]{Primary 05-01}
\keywords{Combinatorics, Discrete Mathematics.}

\maketitle

\setcounter{page}{4}
\tableofcontents

%-----------------------------------------------------------------------------
% Beginning of preface.tex
%-----------------------------------------------------------------------------
%
% AMS-LaTeX 1.2 sample file for a monograph, based on amsbook.cls.
% This is a data file input by chapter.tex.
%%%%%%%%%%%%%%%%%%%%%%%%%%%%%%%%%%%%%%%%%%%%%%%%%%%%%%%%%%%%%%%%%%%%%%%%

\chapter*{Preface}

These are the lecture notes of MAT315 Combinatorial Enumeration, offered in the Monsoon 2024 semester at Ahmedabad University, India. The notes were written down by the TA for the course, Kanak Dhotre, and is as close to the classroom teaching as possible.

There are several very good textbooks in combinatorics, however, the material that I wish to cover in this course is not available in a single source to my liking. So, I decided to make my own notes for this iteration, as well as for any future iterations of this course.

The background required to enroll for the course is very minimal, it is not even mandatory for a student to have done a first course in Discrete Mathematics. So, we introduce several basic concepts along the way and if there is a scope for some digression then we will take it. The course is supplemented by some homework assignments, some of the problems in the text were set in those assignments, and some even appeared in the examinations.

I am thankful to Kanak Dhotre for typing these notes. Any errors that remain are mine. If there are any errors, comments, or corrections, please write to me via email.

\aufm{Manjil Saikia}

%-----------------------------------------------------------------------------
% End of preface.tex
%-----------------------------------------------------------------------------


\mainmatter
\chapter{What is Combinatorics?}
\section{Introduction}
This course aims to delve into the study of discrete mathematical structures, a field which traces its roots back to the 1700s with the work of Leonhard Euler and has gained much attention between the 1960s and 1970s with the advent of computer science. Notably, Euler answered the following question posed by Philip Naude in the year 1741: “In how many ways can the number $50$ be written as a sum of seven different positive integers?”. We shall understand the outline of Euler’s solution to the problem later in this course. A few important personalities (some of whose work we will study eventually) in the subject include Gian Carlo Rota, Donald Knuth, Richard Stanley, Srinivasa Ramanujan, and Pinagala. 

Combinatorics is the science of patterns and arrangements. More concretely, it deals with the study of the existence and the number of arrangements possible for a given mathematical structure. We start our discussion with a few motivating questions which will make our statement clearer.

\begin{question}
	In how many ways can you arrange the elements of the set $[n]:=\{1,2,3,\ldots,n\}$ such that the first entry in the arrangement is an even number?
\end{question}

Notice how when $n$ is an even number we have $n/2$ choices of even numbers to make for the first entry in our arrangement. Once a choice for the said even number is made the remaining $n-1$ choices can be made in $\left( n-1 \right)!$ ways. Hence, in the case where $n$ is an even number we have $n/2 \cdot \left(n-1 \right)!$ possible arrangements. Can you see why we will have $\left( n-1 \right)/2 \cdot \left( n-1 \right)!$ arrangements for the case where $n$ is an odd number?

\begin{question}
    In how many ways can you arrange elements from the set $[n]:=\{1,2,3,\ldots,n\}$ on a grid with $n$ columns and $n$ rows?
\end{question}

Notice how for each one of the $n^2$ spaces we have $n$ choices to make. Hence, there are a total of $\underbrace{n \ \cdots \ n}_{n^2 \text{ times}} = n^{n^2}$ possible arrangements.

\begin{question}
In how many ways can you arrange elements from the set $[n]$ (as defined in the previous two examples) on a grid with $n$ columns and $n$ rows such that each element appears atleast(/exactly) once in each row?
\end{question}

    Since for each row in the grid we have $n!$ possible arrangements and the grid has $n$ rows there are a total of $n\cdot n!$ possible arrangements.

\begin{question}
    How many matrices of order $n\times n$ exist given that the entries must be from the set $\{0,1\}$? 
\end{question}
	Since we have $2$ choices for each one of the $n^2$ entries there are a total of $2^{n^2}$ such matrices.

\begin{question}
\label{q:1.5}
	How many matrices of order $n \times n$ exist given that the entries must be from the set $\{0,1\}$ and each row and column must have exactly one $1$.
\end{question}

Notice how in the first row of our matrix we have $n$ ways to fix the occurrence of $1$. This forces $n-1$ ways to fix the occurrence of $1$ in the second row and so on. Hence, in all there are a total of $n\cdot \left( n-1 \right) \cdots 1 = n!$ such matrices.

\begin{remark}
\cref{q:1.5} can also be re-stated as counting the number of order $n\times n$ matrices which have row-sum and column-sum equal to $1$.    
\end{remark}

With the following definition we shall now look at a generalization of sorts of the kind of matrices we were dealing with in \cref{q:1.5}.

\begin{definition}[Alternating Sign Matrix (ASM)]
	A matrix of order $n\times n$ is called an alternating sign matrix if the following conditions hold:
	\begin{enumerate}
		\item All the entries of the matrix come from the set $\{-1,0,1\}$.
		\item Each row-sum and column-sum is $1$.
		\item The non-zero entries (both row-wise and column-wise) alternate in sign.
	\end{enumerate}
\end{definition}

A result first proved by Doron Zeilberger in the year 1992 states that there are precisely  \[
\prod_{k=0}^{n-1} \frac{\left( 3k+1 \right)!}{\left( n+k \right)!}
\] 
number of ASMs of order $n\times n$. A proof of this result is beyond the scope of these lectures and is mentioned only for the sake of completeness. We shall, however, count ASMs of order $3$ now.

\begin{question}
    How many ASMs of order $3$ exist?
\end{question}
Notice how the set of matrices we counted in \cref{q:1.5} are a subset of the set of ASMs of order $n$ (verify each one of the three defining properties of an ASM). Next, we notice a pattern; an ASM (of any order) can't have a $-1$ in the first row. Why? To the contrary, assume there is an ASM with a $-1$ in the first row. Since the immediate non-zero entry below it must be a $1$, the column sum cannot be $1$ without violating the alternativity condition. A similar argument shows that ASMs cannot have a 
$-1$ in the last row, the first column, or the last column either. This pattern allows us to easily list all ASMs of order $3$.

\[
\underbrace{
\begin{pmatrix}
    1 & 0 & 0 \\
    0 & 1 & 0 \\
    0 & 0 & 1
\end{pmatrix},
\quad
\begin{pmatrix}
    1 & 0 & 0 \\
    0 & 0 & 1 \\
    0 & 1 & 0
\end{pmatrix},
\quad
\begin{pmatrix}
    0 & 1 & 0 \\
    1 & 0 & 0 \\
    0 & 0 & 1
\end{pmatrix},
\quad
\begin{pmatrix}
    0 & 1 & 0 \\
    0 & 0 & 1 \\
    1 & 0 & 0
\end{pmatrix},
\quad
\begin{pmatrix}
    0 & 0 & 1 \\
    1 & 0 & 0 \\
    0 & 1 & 0
\end{pmatrix},
\quad
\begin{pmatrix}
    0 & 0 & 1 \\
    0 & 1 & 0 \\
    1 & 0 & 0
\end{pmatrix}
}_{\text{All the ASMs counted in \cref{q:1.5}.}}
\]

\[
\underbrace{
\begin{pmatrix}
    0 & 1 & 0 \\
    1 & \textcolor{red}{-1} & 1 \\
    0 & 1 & 0
\end{pmatrix}
}_{\text{The only ASM of order $3$ with a negative entry.}}
\]

\section{Counting Principles}
\section{The Pigeon-hole Principle}
\section{The Principle of Inclusion-Exclusion}
\chapter{The Art of Combinatorial Thinking}
\section{Binomial Coefficients}
This sections aims to push-forward our familiarity with binomial coefficients which we have gained from the previous chapter. We do so by outlining proofs of a few standard combinatorial identities. 

\begin{comment}
\begin{question}
3a
\end{question}
You have $2n$ objects
Divide the objects into two classes
Each class has  $n$ objects
Choose  $k$ objetcs from class  $A$
Choose  $n-k$ objects from class  $B$
Choose a distinguished element from the k choices of objects from class  $A$
==
Choose the distinguished element from $A$ first <=> n
Choose the rest of the elements from  $A\cup B \backslash \{x\}$ where  $x$ is the distinguised elements <=> \binom{2n-1}{n-1}

\begin{question}
3b
\end{question}
Think along subsets

\begin{question}
\sum_{i=1}^{n}i(n-1) = \\sum_{i=1}^{n}\binom{i}{2} = \sum_{i=0}^{n-2}\binom{n-i}{2} = \binom{n+1}{3} 
\end{question}
Three numbers out of the $n+1$ choices say $a_{1}<a_{2}<a_{3}$. Now argue for the first three sums based on which one of these is chosen first. 

The first sum corresponds to choosing $a_2$. The second sum corresponds to choosing  $a_{3}$. The third sum corresponds to choosing $a_{1}$.

\begin{question}
	For all $n\geq r\geq 0$ we have 
	 \[
		 \binom{r}{r}+\binom{r+1}{r}+\cdots+\binom{n}{r} = \binom{n+1}{r+1}
	.\] 
\end{question}
Give a proof using lattice paths. 
RHS is the number of lattice paths from $(0,0)\to (r+1,n-r)$. For the LHS what happens before the last E step.

The idea is not always to prove equalities. Infact more often than not we will be interested in proving inequalities
 \begin{question}
	 \binom{2n}{n}<4^n
\end{question}
Convert binary to lattice: 0 to N, 1 to E and so on.

Next we prove the Chu-Vandermonde identity.
\begin{question}
	\binom{2n}{n} = \sum_{k=0}^{n} \binom{n}{k}^2
\end{question}

\begin{question}
	For all $n>0$. \[
		\sum_{k=0}^{n} 2^k\binom{n}{k} = 3^n
	\]
\end{question}

\begin{question}
	For $n>0$  \[
		2\binom{2n-1}{n} = \binom{2n}{n}
	.\] 
\end{question}
Give two proofs. One using Pascals identity. The other by choosing sets.
\end{comment}

\begin{claim}
\[
\sum_{k=1}^{n}k\binom{n}{k}^2 = n\binom{2n-1}{n-1}
\]
\end{claim}
\begin{proof}
Let $J$ be a collection of $2n$ objects which is partitioned into $2$ equal sub-collections $J_1$ and $J_2$. Now, we want to make a choice of $n$ elements from $J$ which includes a distinguished element, say $x$ from $J_1$ (or equivalently $J_2$). Since there are $n$ ways to choose the said distinguished element, and $\binom{2n-1}{n-1}$ ways to choose the rest of the $n-1$ elements, in all we have \[
n\binom{2n-1}{n-1}
\] ways to make the said choice. Alternatively, we can also choose $k$ elements from $J_1$ in $\binom{n}{k}$ ways, $n-k$ elements from $J_2$ in $\binom{n}{n-k}$ ways, and then choose the distinguished element from $J_1$ (or equivalently $J_2$) in $1\leq k\leq n$ ways. Thus far, we have proved
\[
\sum_{k=1}^{n}k\binom{n}{k}\binom{n}{n-k} = n\binom{2n-1}{n-1}.
\]
Since by \cref{r:1.2} \[
\binom{n}{k} = \binom{n}{n-k},
\]
we are done. 
\end{proof}
\begin{claim}
\[
\sum_{k=1}^{n}k\binom{n}{k} = n2^{n-1}
\]
\end{claim}
\begin{proof}
\end{proof}

\begin{claim}
\[
\sum_{i=1}^{n}i(n-1) = \sum_{i=1}^{n}\binom{i}{2} = \sum_{i=0}^{n-2}\binom{n-i}{2} = \binom{n+1}{3} 
\]
\end{claim}
\begin{proof}
\end{proof}

\begin{claim}
For all $n\geq r\geq 0$ we have 
\[
		 \binom{r}{r}+\binom{r+1}{r}+\cdots+\binom{n}{r} = \binom{n+1}{r+1}
.\] 
\end{claim}
\begin{proof}
\end{proof}

\begin{claim}
$\binom{2n}{n}<4^n$
\end{claim}
\begin{proof}
    
\end{proof}

\begin{claim}[Chu-Vandermonde Identity]
\[\binom{2n}{n} = \sum_{k=0}^{n} \binom{n}{k}^2\]
\end{claim}
\begin{proof}
Once again, by \cref{r:1.2}, notice how \[
\sum_{k=1}^n\binom{n}{k}^2 = \sum_{k=1}^{n}\binom{n}{k}\binom{n}{n-k}.
\] Now we argue combinatorially. Let $J$ be a collection of $2n$ objects which is partitioned into $2$ equal sub-collections $J_1$ and $J_2$. Any choice of $n$ elements from $J$ involves choosing $k$ elements from $J_1$ and $n-k$ elements from $J_2$ where $0\leq k\leq n$. 
\end{proof}
\begin{claim}
	For all $n>0$. \[
		\sum_{k=0}^{n} 2^k\binom{n}{k} = 3^n
	\]
\end{claim}
\begin{proof}
\end{proof}
\begin{claim}
For all $n>0$  \[
2\binom{2n-1}{n} = \binom{2n}{n}
.\] 
\end{claim}
\begin{proof}
\end{proof}
\section{Catalan Numbers}
\endinput


\chapter{Generating Functions}




\section{Ordinary Generating Functions}


\section{Exponential Generating Functions}



\endinput
\chapter{Partitions}
A first course in combinatorics typically focuses on two types of partitions: set partitions and integer partitions. We will begin with a brief discussion of set partitions, followed by a more in-depth exploration of integer partitions. 
\section{Set Partitions}
\textcolor{red}{
[This section is a work in progress!!!] Note to scribe:
\begin{enumerate}
    \item Finish proofs of Claim 1.1-1.6.
\end{enumerate}
}
Notice how there are $6$ ways to partition the set $\{1,2,3,4\}$ into $3$ blocks. These are,
    \begin{enumerate}
        \item $[1],[2],[3,4]$
        \item $[1],[2,3],[4]$
        \item $[1,2],[3],[4]$
        \item $[1,4],[2],[3]$
        \item $[1,3],[2],[4]$
        \item $[2,4],[1],[3]$
    \end{enumerate}
This kind of counting is generalized by what are called a Stirling numbers of the $\nth{2}$ kind. More formally,
\begin{definition}
A set partition of a finite set $B$ into $k$ ``blocks'' is a collection of $k$ subsets of $B$ say $B_1,\cdots,B_k$ such that 
\begin{enumerate}
    \item $\bigcup_{i=1}^kB_i = B$,
    \item $B_i\cap B_j=\emptyset$ for all $i\neq j$,
    \item and none of the $B_i$'s are empty. 
\end{enumerate}
\end{definition}
\begin{definition}
If $B=[n]=\{1,\cdots,n\}$ then a Stirling number of the $\nth{2}$ kind $S(n,k)$, is the number of set partitions of $B$ into $k$ blocks. 
\end{definition}
We take $S(0,0)$ to be $1$ by convention. Additionally, the fact that $S(n,n)=1$, $S(n,0)=0$, and $S(n,1)=1$ follow immediately. Per usual we state a few identities concerning these numbers.
\begin{claim}
For all $n,k\geq 0$, with $n\geq k$ we have $S(n,k)=S(n-1,k-1)+kS(n-1,k)$
\end{claim}
\begin{claim}
For all $n,k\geq 0$, with $n\geq k$ we have $S(n+1,k)=\sum_{i=0}^{n}S(n-i,k-1)$
\end{claim}
\begin{claim}
    $S(n,2)=2^{n-1}-1$
\end{claim}
\begin{claim}
    $S(n,n-1)=\binom{n}{2}$
\end{claim}
\begin{claim}
    \[
    S(n,k) = \dfrac{1}{k!}\sum_{j=0}^{k}(-1)^{k-j}\binom{k}{j}j^n
    \]
\end{claim}
\begin{comment}
\begin{proof}
Recall how $\sum_{j=0}^{k}(-1)^{k-j}\binom{k}{j}j^n$ counts the number of surjections from $[n]$ to $[k]$. Hence, it suffices to show that the number of surjections from $[n]$ to $[k]$ is $k!S(n,k)$. 
\end{proof}
\end{comment}
\begin{definition}[Bell Numbers]
The number of all partitions of $[n]$ is called a bell number and is denoted by $B(n)$. More specifically,
\[
B(n) = \sum_{k=1}^nS(n,k)
\]
\end{definition}
We are interested in coming up with a recurrence of $B(n)$ which is independent of any $S(n,k)$s. 
\begin{claim}
    \[
    B(n+1) = \sum_{k=0}^{n}\binom{n}{n-k}B(k)
    \]
\end{claim}
\section{Integer Partitions}
\textcolor{red}{
[This section is a work in progress!!!] Note to scribe: Add the following topics
\begin{enumerate}
    \item Theorem 2.2 and it's relation to the existence of binary repns.
    \item Guessing (and then proving) a Ramanujan congruence.
    \item Three proofs of the EPT.
    \item JTP. 
    \item EPT as a consequence of JTP.
    \item A reccurence for partitions from EPT. 
    \item $q$-Pochammer symbols and a few  re-statements. 
\end{enumerate}
}
Recall how with \cref{d:1.5} we defined integer partitions. The study of these partitions dates back to the work of Leonard Euler in the \nth{18} century, who introduced generating functions to analyze them. Their study was then developed further through the work of mathematicians like Srinivasa Ramanujan and Major MacMachon, who revealed deep arithmetic and combinatorial properties. For instance, if $p(n)$ denotes the number of partitions of $n$, then the celebrated Ramanujan congruences state that
\begin{align*}
P(5n+4)&\equiv 0\mod{(5)} \\
P(7n+5)&\equiv 0\mod{(7)} \\
P(11n+6)&\equiv 0\mod{(11)} \\ 
\end{align*}
In the absence of a closed form (recall \cref{r:1.3}), we are interested in finding the generating function of $p(n)$.
\begin{claim}\[
\prod_{n\geq 1}(1+q^n+q^{2n}+\cdots) = \sum_{n\geq 0}p(n)q^n.
\]
\label{c:2.1P}
\end{claim}
\begin{proof}
We can expand the product on the L.H.S, \[(1+q+q^2+q^3+\cdots)(1+q^2+q^4+q^6+\cdots)(1+q^3+q^6+q^9+\cdots)\cdots,\] out by choosing one term from each factor in all possible ways. If we then collect like terms, the coefficient of $q^k$ will be the number of ways to choose one term from each factor so that the exponents of the said terms sum up to $k$. This is also what the R.H.S counts. For instance $q^3$ can be obtained in the following ways
\begin{enumerate}
    \item Choose $q^3$ from the first bracket and $1$ from every other bracket. This corresponds to the partition $3=1+1+1$.
    \item Choose $q$ from the first bracket, $q^2$ from the second bracket, and $1$ from every other bracket. This corresponds to the partition $3=2+1$. 
    \item Chose $1$ from the first two brackets and $q^3$ from the third bracket. This corresponds to the partition $3=3$. 
\end{enumerate}
\end{proof}
Next, we are interested in looking at a two refinements of partitions which follow quite naturally by the arguments we used in the previous proof.
\begin{claim}
    Let $p_E(n)$ and $p_O(n)$ denote the number of partitions of $n$ into even and odd parts respectively. Then,
    \begin{align*}
    \prod_{n=1,3,5,\cdots}(1+q^n+q^{2n}+\cdots)=\sum_{n\geq 0}p_O(n)q^n \\
    \prod_{n=2,4,6,\cdots}(1+q^n+q^{2n}+\cdots)=\sum_{n\geq 0}p_E(n)q^n 
    \end{align*}
    \label{c:PEPO}
\end{claim}
Next, we state an interesting result, once again due to Euler.
\begin{theorem}[Euler's Gem]
The number of partitions of $n$ into distinct parts, say $p_D(n)$, is the same as the number of partitions of $n$ into odd parts, say $p_O(n)$.
\label{t:Euler'sGem}
\end{theorem}
We will outline two different proofs of the result.
\begin{proof}
By \cref{c:PEPO} we already know that the generating function for $p_O(n)$ is 
\[
\prod_{i=1}^{\infty}\dfrac{1}{1-q^{2i-1}}.
\]
It is also clear (by arguments similar to the ones which were involved in the proof of $\cref{c:2.1P}$) that the generating function for $p_D(n)$ is 
\[
(1+q)(1+q^2)(1+q^3)\cdots = \prod_{i=1}^\infty(1+q^i).
\]
Now to complete the proof it suffices to show that the two are equal. To this end, notice how 
\begin{align*}
    \prod_{i=1}^\infty(1+q^i) &= \prod_{i=1}^\infty (1+q^i)\dfrac{1-q^i}{1-q^i} \\
    &= \prod_{i=1}^\infty \dfrac{1-q^{2i}}{1-q^i} \\\
    &= \dfrac{(1-q^2)(1-q^4)(1-q^6)\cdots}{(1-q)(1-q^2)(1-q^3)(1-q^4)\cdots} \\
    &= \prod_{i=1}^\infty \dfrac{1}{1-q^{2i-1}}
\end{align*}
\end{proof}
In the spirit of giving a combinatorial proof, we want to setup a bijection called Glashier's bijection between the two types of partitions.
\begin{proof}
First we setup a map which sends a partition into odd parts to a partition into distinct parts. The most natural thing to do is to merge any pairs of repeating parts into one part of double the size. We can repeat this procedure until all the parts are distinct. For instance $3+3+3+1+1+1+1\to (3+3)+3+(1+1)+(1+1)= 6+3+2+2 = 6+3+(2+2) = 6+4+3$. Next, we setup a map which sends a partition into distinct parts to a partition into odd parts. Once again, the most natural thing to do is split every occurrence of an even part into two equal parts. We can repeat this procedure until all the parts are odd. For instance $6+4+3\to (3+3)+(2+2)+3\to 3+3+3+1+1+1+1$.
\end{proof}
We state a generalization of Euler's gem now. 
\begin{theorem}
The number of partitions where no part appears $d$ or more times is the same as the number of partitions where no part is divisible by $d$.
\label{t:glashier}
\end{theorem}
\begin{proof}
Let $p_1(n)$ denote the number of partitions of $n$ with no parts divisible by $d$. Let $p_2(n)$ denote the number of partitions of $n$ where no part appears $d$ or more times. Notice how the generating function for $p_1(n)$ is given by \[
\sum_{n=0}^{\infty}p_1(n)q^n = \prod_{n=1, d\nmid}^{\infty}\dfrac{1}{1-q^n}
\]
and that of $p_2(n)$ is given by 
\begin{align*}
\sum_{n=0}^{\infty}p_2(n)q^n &= \prod_{n=1}^{\infty}\dfrac{1-q^{dn}}{1-q^n} \\
&= \dfrac{1-q^d}{1-q}\dfrac{1-q^{2d}}{1-q^2}\cdots\dfrac{1-q^{kd}}{1-q^k}\cdots.
\end{align*}
Finally, each term in the numerator cancels with the corresponding multiple of $d$ in the denominator and we are left with the generating function for $p_1(n)$. This completes the proof.
\end{proof}
\begin{remark}
One might ask in what sense is \cref{t:glashier} a generalization of \cref{t:Euler'sGem} To see this, notice how setting $d=2$ returns the Euler's gem.  
\end{remark}
As it turns out many partition identities are best explained using pictures. To this end we introduce a tool called Ferrer's diagrams. In the said diagram the parts of a partition are shown as rows of dots/squares. 
\begin{figure}[H]
    \centering
    \includegraphics[width=0.5\linewidth]{Images/Figure17.png}
    \caption{The Ferrer's diagram for $7+5+3+2+2+1$}
\end{figure}
It is clear that height of the Ferrer's diagram corresponds to the number of parts in the partition and that the width corresponds to the size of the largest part in the partition. To this end, consider the following result. 
\begin{theorem}
The  number of partitions of $n$ into atmost $i$ parts, each of which is atmost $j$ is the same as the number of partitions of $n$ into atmost $j$ parts, each of which is atmost $i$. 
\end{theorem}
\begin{proof}
Let $\pi$ a partition of $n$ into atmost $i$ parts each of which is atmost $j$. The Ferrer's diagram corresponding to $\pi$ now has height atmost $i$ and width atmost $j$. Flipping the said diagram about it's main diagonal results in a diagram which has height atmost $j$ and width atmost $i$. Finally, the partition corresponding to this flipped diagram (called the conjugate partition) is a partition of $n$ into atmost $j$ parts, each of which is atmost $i$. 
\end{proof}
\begin{figure}[H]
    \centering
    \includegraphics[width=0.8\linewidth]{Images/Figure18.png}
    \caption{An example of the conjugate of $7+5+3+2+2+1$}
\end{figure}
Another useful pictorial construct is that of a Durfee square. We define the said square to be the largest one which can fit in the Ferrer’s diagram corresponding to a partition.
\begin{figure}[H]
    \centering
    \includegraphics[width=0.5\linewidth]{Images/Figure19.png}
    \caption{The Durfee square of order $3$ in the partition $7+5+3+2+2+1$}
    \label{DurfeeExample}
\end{figure}
(Keeping in mind \cref{DurfeeExample}) notice how
\begin{enumerate}
    \item Every non-empty partition has a Durfee square of order $k$ (if nothing, you always have the Durfee square of order $k=1$).
    \item Partitions corresponding to the triangle above the Durfee square are the ones where each part is atmost $k$.
    \item Partitions corresponding to the triangle below the Durfee square are the ones where number of parts are atmost $k$. 
\end{enumerate}
With these three observations at hand, an identity presents itself almost immediately. Namely,
\begin{claim}
    \[
    \prod_{i=1}^{\infty}\dfrac{1}{1-q^i} = \sum_{i=0}^{\infty}\dfrac{q^{i^2}}{(1-q)^2(1-q^2)^2\cdots(1-q^i)^2}
    \]
\end{claim}
\begin{theorem}[Euler's Pentagonal Number Theorem]
\begin{align*}
\prod_{i=1}^{n}(1-q^i) &= \sum_{k=-\infty}^{\infty}(-1)^k q^{\dfrac{k(3k-1)}{2}} \\
&= 1+\sum_{k=1}^{\infty}(-1)^kq^{\dfrac{k(3k-1)}{2}}+\sum_{k=-\infty}^{-1}(-1)^kq^{\dfrac{k(3k-1)}{2}} \\
&= 1+\sum_{k=1}^{\infty}(-1)^kq^{\dfrac{k(3k-1)}{2}}+\sum_{k=1}^{\infty}(-1)^kq^{\dfrac{k(3k+1)}{2}} \\
&= 1+\sum_{k=1}^{\infty}(-1)^k \left(q^{\dfrac{k(3k-1)}{2}}+q^{\dfrac{k(3k+1)}{2}}\right) \\
\end{align*}
\end{theorem}
%Give a proof using the bi-variate generating 

\endinput

\chapter{\texorpdfstring{$q-$}-Combinatorics}
%\section{Dyck Paths Revisited}
%\section{Motzkin and Schr\"oder Paths}
%\section{Non-intersecting lattice paths}
\section{\texorpdfstring{$q-$}-analogs}
The idea here is to count objects with weights associated with them. For instance in a lattice path, one might be interested in assigning the number of blocks spanned below the said path and/or the number of east steps below the main diagonal. We shall start our discussion by counting all possible (what are called) inversions on the set $[n]$.

\begin{definition}[Inversion]
Let $\sigma$ be a bijection on $[n]$. An inversion of $\sigma$ is a tuple $(\sigma(i),\sigma(j))$ such that $1\leq i<j\leq n$ and $\sigma(i)>\sigma(j)$.
\end{definition}
As an example, we count the number of inversions on the set $[3]$
\begin{center}
\begin{tabular}{|c|c|c|}
\hline
\textbf{Permutation} & \textbf{Inversions} & \textbf{Remark} \\
\hline
$(123)$ & No Inversions & Trivial \\
\hline
$(132)$ & $(32)$ & $2<3$ but $3>2$ \\
\hline 
$(213)$ & $(21)$ & $1<2$ but $2>1$ \\
\hline
$(231)$ & $(21),(31)$ & Same As Above \\
\hline
$(312)$ & $(31),(32)$ & Same As Above \\
\hline
$(321)$ & $(32),(31),(21)$ & Notice a pattern here when the permutation is ``decreasing''\\\hline
\end{tabular}
\end{center}
\begin{claim}
    If $I_n$ denote the number of inversions on $[n]$, then $I_n = \binom{n}{2}\dfrac{n!}{2}$
\end{claim}
\begin{proof}
We attempt to come up with a recursive formula first. Let $I_n$ be known. Now consider the addition of a new symbol (indexed by $n+1$). Notice how the symbol $n+1$ can be placed at a total of $n+1$ places. Placing $n+1$ at the $\nth{1}$ position grants $n$ new inversions. Similarly, placing $n+1$ at the $\nth{2}$ position grants $n-1$ new inversions, and so on. Adding all of these together grants 
\begin{align*}
I_{n+1}&=\underbrace{n!n + I_n}_{\substack{I_n \text{ previously counted inversions} \\ + n\text{ new ones for each one of the } n! \text{ permutations} }} + n!(n-1)+I_n + \cdots + n!(1)+I_n + n!(0)+I_n \\
	       &=n!\left(n+(n-1)+(n-2)+\cdots+2+1+0\right)+(n+1)I_n \\
	       &=n!\left(\dfrac{n(n+1)}{2}\right)+(n+1)I_n \\
	       &=n! \left(\begin{array}{c}n+1\\ 2\end{array}\right) + (n+1)I_n
\end{align*}
From here on end, the proof can also be completed using the method of induction. However, this is not the way we want to go about the proof. Notice how the formula we've obtained is a recursion which can be solved using more than one technique to arrive at an explicit expression. We start off by using the method of back-substitution and re-write the obtained recurrence as
\begin{align*}
	I_n = nI_{n-1}+(n-1)! \left(\begin{array}{c}n\\ 2\end{array}\right) = nI_{n-1}+n!\dfrac{n-1}{2}
\end{align*}
Next, we set $n\to n-1$ and $n\to n-2$ in the obtained equations to arrive at
\begin{align*}
	I_{n-1} = (n-1)I_{n-2} + (n-1)!\dfrac{n-2}{2} \\
	I_{n-2} = (n-2)I_{n-3} + (n-2)!\dfrac{n-3}{2} 
\end{align*}
Making appropriate substitutions gives
\begin{align*}
	I_n &= n\left\{(n-1)I_{n-2}+(n-1)!\left(\dfrac{n-2}{2}\right)\right\}+n!\dfrac{n-1}{2} \\
	    &= n(n-1)I_{n-2}+\dfrac{n!}{2}\left\{(n-2)+(n-1) \right\}.
\end{align*}
More specifically,
\begin{align*}
	I_n = n(n-1)(n-2)I_{n-3} + \dfrac{n!}{2}\left\{(n-3)+(n-2)+(n-1)\right\}.
\end{align*}
It is easy to notice a pattern, that is, after $k$-many such back-substitutions we get
\begin{align*}
	I_n = n(n-1)(n-2)\cdots(n-k+1)I_{n-k} + \dfrac{n!}{2}\left\{(n-1)+(n-2)+\cdots+(n-k)\right\}.
\end{align*}
Setting $n\to n-k$ gives
\begin{align*}
	I_{n-k}=(n-k)I_{n-k-1} + (n-k)!\dfrac{n-k-1}{2}.
\end{align*}
Once again, making appropriate substitutions we get.
\begin{align*}
I_n &= n(n-1)\cdots(n-k+1)(n-k)I_{n-k-1}\\ &\quad +\dfrac{n!}{2}\left\{(n-k-1)+(n-1)+(n-2)+\cdots+(n-k)\right\}.
\end{align*}
Finally, to see why the claim is true it suffices to set $k\to n-1$ because then we would have 
\begin{align*}
	I_n = n(n-1)(n-2)\cdots (2)(1)I_{0} + \dfrac{n!}{2}\left\{ (n-1)+(n-2)+(n-3)+\cdots+1\right\}.
\end{align*}
Since $I_0=0$, we get
\begin{align*}
	I_n = \dfrac{n!}{2}\left(\begin{array}{c}n\\ 2\end{array}\right) 
\end{align*}
as required. 
\end{proof}
Next, we give a combinatorial proof. 
\begin{proof}
For an $n$ given to us, consider all the $n!$ possible permutations on the set $[n]$ arranged in pairs like $(\sigma(1)\sigma(2)\cdots\sigma(n)), \underbrace{(\sigma(n),\sigma(n-1),\cdots,\sigma(1))}_{\text{Called the mate of } \sigma}$. This arrangement separates the $n!$ permutations into $n!/2$ pairs. Now, by the following observations we are done.
\begin{enumerate}
	\item If $(\sigma(i),\sigma(j))$ is an inversion of $\sigma$, then it's not an inversion of $\sigma$'s mate.
	\item Each one of the $\left(\begin{array}{c}n\\ 2\end{array}\right)$ pairs is an inversion exactly once in each couple.
\end{enumerate}
\end{proof}
Corresponding to a given $n$ we know that there are $n!$ possible permutations on the set $[n]$. In the formal variable $q$, we define the inversion polynomial on $[n]$ as \[\sum_{\sigma\in \text{Bijections on }[n]}q^{\text{inv}(\sigma)}\] where $\text{inv}(\sigma)$ denotes the number of inversions of $\sigma$.  As an example, we compute the inversion polynomial on the set $[3]$.
\begin{center}
\begin{tabular}{|c|c|c|}
\hline
\textbf{Permutation} & \textbf{Inversions} & $q^{\text{inv}(\sigma)}$ \\
\hline
$(123)$ & No Inversions & $q^0$ \\
\hline
$(132)$ & $(32)$ & $q^1$ \\
\hline 
$(213)$ & $(21)$ & $q^1$ \\
\hline
$(231)$ & $(21),(31)$ & $q^2$ \\
\hline
$(312)$ & $(31),(32)$ & $q^2$\\
\hline
$(321)$ & $(32),(31),(21)$ & $q^3$\\
\hline
\end{tabular}
\label{tab:S3Example}
\end{center}
It is clear that the inversion polynomial corresponding to $[3]$ is given by $1+q+q+q^2+q^2+q^3=(1+q)(1+q+q^2)$.
\begin{question}
What is the inversion polynomial corresponding to $[n]$ for an arbitrary choice of $n$?
\end{question}
\begin{solution}
We know that the inversion polynomial for $[1]$ is $q^0=1$, for $[2]$ is $q^0+q^1=1+q$, for $[3]$ as shown above is $(1+q)(1+q+q^2)$. One might be tempted to (correctly) assume that the inversion polynomial for $[n]$ is $(1+q)(1+q+q^2)\cdots(1+q+q^2+\cdots+q^{n-1})$. Notice how the addition of a symbol indexed by $4$ in the first place raises the degree by $3$.
\begin{center}
\begin{tabular}{|c|c|c|}
\hline
\textbf{Permutation} & \textbf{Inversions} & $q^{\text{inv}(\sigma)}$ \\
\hline
$(4123)$ &$(41),(42),(43)$ & $q^{0+3}$ \\
\hline
$(4132)$ & $(41),(43),(42),(32)$ & $q^{1+3}$ \\
\hline 
$(4213)$ & $(42),(41),(43),(21)$ & $q^{1+3}$ \\
\hline
$(4231)$ & $(42),(43),(41),(21),(31)$ & $q^{2+3}$ \\
\hline
$(4312)$ & $(43),(41),(42),(31),(32)$ & $q^{2+3}$\\
\hline
$(4321)$ & $(43),(42),(41),(32),(31),(21)$ & $q^{3+3}$\\
\hline
\end{tabular}
\end{center}
Hence, the sum corresponding to the addition of a symbol indexed by $4$ at the first position becomes \[q^3\underbrace{((1+q)(1+q+q^2))}_{\text{Inversion polynomial of } S_3}.\] Similarly, the addition of this symbol at the second place raises the degree by $2$, the addition of this symbol at the third raises the degree by $1$, and so on. Finally, we get 
\begin{align*}
	\sum_{\sigma\in \text{Bijections on }[4]} q^{\text{inv}(\sigma)}&=q^3((1+q)(1+q+q^2))+ q^2((1+q)(1+q+q^2))\\ &\quad +q^1((1+q)(1+q+q^2))+q^0((1+q)(1+q+q^2)) \\
						   &=(1+q)(1+q+q^2)(1+q+q^2+q^3)
\end{align*}
This allows us to conclude that \[
\sum_{\sigma\in \text{Bijections on }[n]}q^{\text{inv}(\sigma)} = (1+q)(1+q^2)\cdots (1+q^{n-1})
\]
\end{solution}
Polynomials of the form involved in our solution keep coming up in the study of $q$-combinatorics. For this reason, we introduce some notation for succinct writing (amongst other reasons which will be explained soon).
\begin{definition}[$q$-analogue of numbers]
For a real number $n$, we denote it's $q$-analogue by \[[n]_q:=\begin{cases}\dfrac{1-q^n}{1-q} \ &\text{ if } q\neq 1 \\ n \ &\text{ if } q=1 \end{cases}.\]
\end{definition}
The following definition now follows quite naturally.
\begin{definition}[$q$-analogue of factorials]
For a real number $n$, we denote the $q$-analogue of its factorial by 
\[n!_q=[1]_q[2]_q\cdots [n]_q\].
\label{d:q_fact}
\end{definition}
\begin{remark}
The introduction of \cref{d:q_fact} allows us to conclude that the inversion polynomial corresponding to $[n]$ is $n!_q$. 
\end{remark}
In fact, yet another definition follows quite naturally.
\begin{definition}[$q$-analogue of binomial coefficients]
    For appropriate choices of $n$ and $k$, we denote the $q$-analogue of $\binom{n}{k}$ by 
    \[
    \binom{n}{k}_q = \dfrac{n!_q}{(n-k)!_q k!_q}.
    \]
    \label{d:qBin}
\end{definition}
However, a combinatorial interpretation of \cref{d:qBin} is not immediately clear. To this end, consider the following problem.
\begin{question}
Let $S(k,n-k)$ denote the set of all $n$-bit sequences with $k$ zeros and $n-k$ ones. What is the inversion polynomial corresponding to $S(k,n-k)$?
\end{question}
The following table lists all the possible $4$-bit sequences with $2$-zeros along with their contributions to the inversion polynomial.
\begin{center}
	\begin{tabular}{|c|c|c|}
		\hline
		$(0011)$ & $0$ inversions & $q^0$ \\
		\hline
		$(0101)$ & $(10)$ once & $q^1$ \\
		\hline
		$(0110)$ & $(10)$ twice & $q^2$ \\
		\hline
		$(1001)$ & $(10)$ twice & $q^2$ \\
		\hline
		$(1010)$ & $(10)$ thrice & $q^3$ \\
		\hline
		$(1100)$ & $(10)$ four times & $q^4$ \\
		\hline
	\end{tabular}
\end{center}
\raggedbottom
From the table, it is clear that the inversion polynomial corresponding to $S(k,n-k)$ is given by
\begin{align*}
	\sum_{\sigma}q^{\text{inv}(\sigma)} &= 1+q+2q^2+q^3+q^4 \\
					    &= (1+q+q^2)(1+q^2) \\
					    &= [3]_q (1+q^2) \\
					    &= [3]_q (1+q^2)\dfrac{1+q}{1+q} \\
					    &= [3]_q \dfrac{(1+q+q^2+q^3)}{1+q} \\
					    &= [3]_q \dfrac{[4]_q}{[2]_q} \\
					    &= \dfrac{[4]_q[3]_q[2]_q[1]_q}{[2]_q[1]_q[2]_q[1]_q} \\
					    &= \dfrac{4!_q}{2!_q2!_q}\\
					    &= \dfrac{4!_q}{2!_q(4-2)!_q}.
\end{align*}
More generally, we have the following theorem.
\begin{theorem}
Let $S(k,n-k)$ denote the set of all $n$-bit sequences of $k$ zeros and $n-k$ ones. Then the inversion polynomial corresponding to $S(k,n-k)$ is \[\sum_{\sigma\in S(k,n-k)} q^{\text{inv}(\sigma)}=\left(\begin{array}{c}n \\ k \end{array}\right)_q = \sum_{j=0}^{k(n-k)}c_j(k,n-k)q^j\] where $c_j(k,n-k)$ counts the number of $n$-bit string with exactly $k$ zeros and $j$ inversions.
\label{t:FQB}
\end{theorem}
\begin{claim}[A $q$-analogue of Pascal's identity]
\[
    \binom{n+1}{k}_q = \binom{n}{k}_q + q^{n-k+1}\binom{n}{k-1}_q
\]
\label{c:q_Pascal}
\end{claim}
\begin{proof}
We give a double counting argument. Notice how the L.H.S is the inversion polynomial corresponding to $S(k,n+1-k)$. It is also true that every $n+1$-bit sequence in $S(k,n+1-k)$ either ends with a $1$ in which case it is not inverted with any of the preceding symbols - this explains the first term of the R.H.S, or ends with a $0$ in which case it is inverted with all the $n-k+1$-many $1$s - this explains the second term of the R.H.S. 
\end{proof}
Given our introduction to $q$-binomial coefficients, it is natural to ask if there is such a thing as $q$-multinomial coefficients as well. To this end, consider,
\begin{definition}
The multinomial coefficient corresponding to non-negative $k_1,\ldots,k_m$ adding up to $n$ is denoted by
\[
\binom{n}{k_1,\ldots,k_m} := \dfrac{n!_q}{k_1!_q\cdots k_m!_q}.
\]
\end{definition}
As one might expect $\binom{n}{k_1,\ldots,k_m}$ is the inversion polynomial corresponding to $S_n(k_1,\ldots,k_m)$, the set of all $n$-length permutations with $k_i$-many $i$s ($i=1,\ldots,m$). We shall omit the proof, however a proof by induction is not too difficult to work out. Once again, it is also natural to ask if there is such a thing as the $q$-binomial theorem. Infact there are several of them. We state one of them here. 
\begin{theorem}
\[
\prod_{i=1}^{n}(1+xq^i) = \sum_{i=0}^{n}\binom{n}{i}_q q^{i(i+1)/2}x^i
\]
\end{theorem}
\begin{proof}
Notice how $\prod_{i=1}^{n}(1+xq^i)$ can be written as \[
\sum_{i=0}^{n}a_i(q)x^i
\]
where $a_i(q)$ is the generating function of partitions into distinct parts with exactly $i$ part, where each part is $\leq n$. With this observation at hand it suffices to show that $a_i(q)$ is 
\[
\binom{n}{i}_q q^{i(i+1)/2}=\binom{n}{i}q^{1+2+3+\cdots+i}.
\]
To this end let $\lambda=\lambda_1+\cdots+\lambda_i$ be a partition into distinct parts and consider the partition into exactly $i$ parts where each part is $\leq n-i$ which is constructed by removing $i$ from the first part, $i-1$ from the the second part, and so on. More specifically consider $\lambda' = (\lambda_1-i)+(\lambda_2-(i-1)+\cdots+\lambda_i-1)$. The generating function for such partitions, as we know, is \[
\binom{n-i+1}{i}_q = \binom{n}{i}_q.
\]
This completes the proof.
\end{proof}
\section{\texorpdfstring{$q-$}-Counting of lattice paths}
We are interested, once again, in the counting of lattice paths, but with weights this time. More specifically to each step in a lattice path we assign the number of unit blocks right below it as it's weight. 
\begin{figure}[H]
    \centering
    \includegraphics[width=0.65\linewidth]{Images/Figure27.png}
    \caption{A lattice path with weight $q^0q^1q^1q^3=q^5$.}
    \label{f:F2L}
\end{figure}
We want to give a $q$-analogue of the double counting argument we used in \cref{q:1.8} to come up with a useful recurrence. 
\begin{figure}[H]
    \centering
    \includegraphics[width=0.65\linewidth]{Images/Figure28.png}
    \caption{}
    \label{fig:Cayley_Rec}
\end{figure}
Refer to \cref{fig:Cayley_Rec} and notice how if a lattice path starts with an $E$-step then we only have to care about the weights due the ``smaller'' path over the remaining blocks (colored blue). On the other hand, if a lattice path starts with an $N$-step then we first row (colored white) always adds a weight of $1$ block and we only have to take care of the ``smaller'' lattice path over the remaining blocks (colored red). Putting \cref{c:q_Pascal} and our observation together allows us to deduce the following result due to George Polya.
\begin{theorem}
Let $A_{n,k}(r)$ denote the number of $n$-step lattice paths from $(0,0)$ to $(k,n-k)$ which span $r$ unit blocks under them. Then 
\[
\binom{n}{k}_q = \sum_{r=0}^{k(n-k)}A_{n,k}(r)q^r
\]
\end{theorem}
It is not difficult to see that every lattice path corresponds to a Ferrer's diagram and hence a partition. For instance, the lattice path in \cref{f:F2L} corresponds to the partition $3+1+1$ of $5$. To this end, consider the following theorem due to Cayley.
\begin{theorem}
Let $p_{i,j}\left( n \right)$ denote the number of partitions of $n$ into atmost $i$ parts each of which is atmost $j$. Then \[
	\sum_{n=0}^{ij}p_{i,j}\left( n \right) q^n = \binom{i+j}{i}_q 
\]     
\end{theorem}
\begin{proof}
Let $c_{i,j}\left( n \right)$ denote the number of sequences with $i$-many $0$s and $j$-many $1$s having exactly $n$-many inversions. We know that
\begin{align*}
	\binom{i+j}{i}_q = \sum_{n=0}^{ij}c_{i,j}\left( n \right)q^n.
\end{align*}
Also, if $A_{i,j}\left( n \right)$ denotes the number of lattice paths from $\left( 0,0 \right)$ to $\left( i,j \right)$ we also know that
\begin{align*}
	\binom{i+j}{i}_{q} = \sum_{n=0}^{ij}A_{i,j}\left( n \right)q^n.
\end{align*}
Hence, to prove our claim it suffices to construct a bijection between $p_{i,j}\left( n \right)$ and $A_{i,j}\left( n \right)$ and/or $c_{i,j}\left( n \right)$. To this end, consider the following partition \[
	n = \pi_{1}+\pi_{2}+\cdots+\pi_{i}
\]
where $\pi_{k}\leq j$ for all possible choices of $k$. For every such partition, it is possible to construct a sequence of $j$-many $0$s and $i$-many $1$s, say $\sigma$, which has exactly $\pi_{1}$-many $0$s followed by the first occurrence of $1$ in $\sigma$, $\pi_{2}$-many $0$s followed by the second occurrence of $1$ in $\sigma$, and so on, all the way up to $\pi_{i}$-many $0$s followed by the last occurrence of $1$ in $\sigma$. This gives a bijection between $p_{i,j}\left( n \right)$ and $c_{j,i}\left( n \right)$. Next, we give a bijection between $p_{i,j}\left( n \right)$ and $A_{i,j}\left( n \right)$. Let $\sigma \in S\left(j,i\right)$ be the sequence of $j$-many $0$s and  $i$-many $1$s corresponding to the partition of $n$ considered above. Now, setting the occurrence of a $1$ in $\sigma$, and the occurrence of a $0$ in $\sigma$ to an $N$ move, and an $E$ move respectively in the Ferrers diagram of $\pi_{1}+\cdots+\pi_{i}$ gives a path from $\left( 0,0 \right)$ to $\left( i,j \right)$ which spans exactly $n$ blocks under it.
\end{proof}
We conclude this section by introducing two different $q$-analogues of Catalan numbers. Before doing so we introduce a new statistic on permutations. 
\begin{definition}
Let $\sigma$ be a permutation on $[n]$. An integer $1\leq i\leq n-1$ is called a descent of $\sigma$ if $\sigma(i)>\sigma(i+1)$. 
\end{definition}
The set of all descents of a permutation is called it's descent set. For instance the descent set of $(613524)$ is $\{1,4\}$. Finally, the major index of a permutation is the sum of all the elements in the descent set. These definitions might seem unmotivated. However, consider the following result which we state without a proof (one proof is outlined as an exercise in Assignment 3). 
\begin{theorem}
Let $S_n$ denote the set of all permutations on $[n]$. Let $\text{maj}(\sigma)$ denote the major index of a permutation $\sigma$ in $S_n$. Then
\[
    \sum_{\sigma\in S_n} q^{\text{maj}(\sigma)} = n!_q.
\]
\end{theorem}
With this background at hand, we are ready to introduce a $q$-analogue of Catalan numbers first given by MacMahon.
\begin{theorem}
Let $L^+$ denote the set of all lattice paths from $(0,0)$ to $(n,n)$ which never go below the main diagonal. For an arbitrary choice of $\pi\in L^+$, let $\sigma(\pi)$ denote the $2n$-length sequence obtained by replacing each occurrence of an $N$ with $0$ and each occurrence of an $E$ with a $1$. Then 
    \[
    \sum_{\pi\in L^+} q^{\text{maj}(\sigma(\pi))} = \dfrac{1}{[n+1]_q}\binom{2n}{n}_q
    \]
\label{t:qCat1}
\end{theorem}
We omit the proof and present yet another natural $q$-analogue of Catalan numbers, one which satisfies a recurrence relation similar to \cref{t:segner}. Per usual, we start with a definition.
\begin{definition}
    Let $L^+$, $\pi$ and $\sigma(\pi)$ be as defined in the setting of \cref{t:qCat1}. By $\text{area}(\pi)$ we refer to sum of the components of the vector obtained by counting the number of complete unit squares to the right at the occurrence of each $N$ step in $\pi$.
\end{definition}
\begin{figure}[H]
    \centering
    \includegraphics[width=0.5\linewidth]{Images/Figure31.png}
    \caption{An example of a path in $L^+$ with area vector $(1,1,2,1,2)$ and area $1+1+2+1+2=7$.}
    \label{fig:enter-label}
\end{figure}
Now, by a result due to Carlitz and Riordan we have.
\begin{theorem}
    If we set $C_n(q)=\sum_{\pi\in L^+}q^{\text{area}(\pi)}$, then \[
    C_n(q) = \sum_{k=1}^n q^{k-1} C_k(q) C_{n-k}(q).
    \]
\end{theorem}
\begin{proof}
The proof follows from the following trick. We decompose the path \( \pi \) by identifying the point of first return to the main diagonal. Suppose the said point is \( (k, k) \). Then, the segment of \( \pi \) from \( (0, 1) \) to \( (k-1, k) \), when treated as an element of the set of all lattice paths from $(0,0)$ to $(k-1,k-1)$ which remain above the main diagonal, has an area that is \( k - 1 \) less than when the same segment is treated as part of $L^+$.
\end{proof}
\endinput


\chapter{A Combinatorial Miscellany}







\endinput


\backmatter
\include{biblio}
%\include{index}
\end{document}

%-----------------------------------------------------------------------
% End of chapter.tex
%-----------------------------------------------------------------------
